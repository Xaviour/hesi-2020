Simplicial sets arise from many topics in mathematics,
and they have a wide range of applications in various branches of mathematics.
For our purpose, they will be used as a model for $\infty$-categories,
as well as ordinary categories.
We will see how this is done in this section.

\subsection{Definition and examples}

Before giving the actual definition,
let us look at some examples of simplicial sets.

\begin{example}
    Let $X$ be a simplicial complex (as in topology),
    together with an ordering of vertices for each simplex $\sigma$,
    such that the inclusion of a face of $\sigma$ into $\sigma$
    preserves the ordering of vertices.
    Let $X_n$ denote the set of $n$-simplices of $X$, which may be degenerate.
    Then we have a series of maps
    \[\begin{tikzcd}
        \cdots\ar[r,shift left=3ex]\ar[r,shift left=1ex]\ar[r,shift right=1ex]\ar[r,shift right=3ex] &
        X_2\ar[l,shift left=2ex]\ar[l]\ar[l,shift right=2ex]\ar[r,shift left=2ex]\ar[r]\ar[r,shift right=2ex] &
        X_1\ar[l,shift left=1ex]\ar[l,shift right=1ex]\ar[r,shift left=1ex]\ar[r,shift right=1ex] &
        X_0\rlap{\ ,}\ar[l]
    \end{tikzcd}\]
    % \xymatrix{ \cdots\ar@<3ex>[r]\ar@<1ex>[r]\ar@<-1ex>[r]\ar@<-3ex>[r] & X_2\ar@<2ex>[l]\ar[l]\ar@<-2ex>[l]\ar@<2ex>[r]\ar[r]\ar@<-2ex>[r] & X_1\ar@<1ex>[l]\ar@<-1ex>[l]\ar@<1ex>[r]\ar@<-1ex>[r] & X_0\rlap{\ ,}\ar[l] }
    where $X_n$ has $(n+1)$ maps to $X_{n-1}$, called the \emph{face maps},
    defined by taking the $(n+1)$ faces of an $n$-simplex.
    $X_n$ also has $(n+1)$ maps to $X_{n+1}$, called the \emph{degeneracy maps},
    defined by regarding an $n$-simplex as a degenerate $(n+1)$-simplex.
    This structure is called a \emph{simplicial set}. \varqed
\end{example}

\begin{example}
    Let $X$ be a topological space, and denote
    \[\operatorname{Sing}(X)_n:=\Hom_{\cat{Top}}(\Delta^n,X),\]
    where $\Delta^n$ denotes the standard $n$-simplex in topology.
    Then $\operatorname{Sing}(X)$ is a simplicial set,
    having the same structure as in the previous example.
    This construction is used to define singular (co)homology in algebraic topology. \varqed
\end{example}

Now we will give the formal definition of a simplicial set.

\begin{definition}
    The category $\bfDelta$ is defined as follows.
    \begin{itms}
        \item Its objects are the sets $[n]:=\{0,\dotsc,n\}$ for all integers $n\geq0$.
        \item The hom-set $\Hom_\bfDelta([m],[n])$ consists of all maps from $[m]$ to $[n]$
              preserving the order $\leq$.
    \end{itms}
\end{definition}

In the category $\bfDelta$, there are two special classes of morphisms.
\begin{itms}
    \item There are $n$ maps from $[n]$ to $[n-1]$,
    denoted by $d^i$ $(0\leq i\leq n-1)$,
    defined by merging the elements $i$ and $i+1$ in $[n]$.
    These are called the \term{coface maps}.
    \item There are $(n+2)$ maps from $[n]$ to $[n+1]$,
    denoted by $s^i$ $(0\leq i\leq n+1)$,
    defined by skipping the element $i$ in $[n+1]$.
    These are called the \term{codegeneracy maps}.
\end{itms}
In fact, all morphisms in $\bfDelta$
can be written as a composition of these coface and codegeneracy maps.
These maps form a diagram
\[\begin{tikzcd}
    \relax[0]\ar[r] &
    \relax[1]\ar[l,shift left=1ex]\ar[l,shift right=1ex]\ar[r,shift left=1ex]\ar[r,shift right=1ex] &
    \relax[2]\ar[l,shift left=2ex]\ar[l]\ar[l,shift right=2ex]\ar[r,shift left=2ex]\ar[r]\ar[r,shift right=2ex] &
    \cdots\ar[l,shift left=3ex]\ar[l,shift left=1ex]\ar[l,shift right=1ex]\ar[l,shift right=3ex]
\end{tikzcd}\]
% \xymatrix{ [0]\ar[r] & [1]\ar@<1ex>[l]\ar@<-1ex>[l]\ar@<1ex>[r]\ar@<-1ex>[r] & [2]\ar@<2ex>[l]\ar[l]\ar@<-2ex>[l]\ar@<2ex>[r]\ar[r]\ar@<-2ex>[r] & \cdots\ar@<3ex>[l]\ar@<1ex>[l]\ar@<-1ex>[l]\ar@<-3ex>[l] }
in the category $\bfDelta$.

\begin{definition}
    A \term{simplicial set} is a functor from $\bfDelta\op$ to $\cat{Set}$,
    i.e.\ a contravariant functor from $\bfDelta$ to $\cat{Set}$.
    We denote the category of simplicial sets by
    \[\cat{sSet}:=\operatorname{Fun}(\bfDelta\op,\cat{Set}).\]
    More generally, for any category $\cat C$,
    a \term{simplicial object} in $\cat C$ is a functor from $\bfDelta\op$ to $\cat{C}$,
    and a \term{cosimplicial object} in $\cat C$ is a functor from $\bfDelta$ to $\cat{C}$.
\end{definition}

Let $X$ be a simplicial set.
The set $X_n:=X([n])$ is called the set of \term{$n$-simplices} of $X$.
The maps 
\[ d_i\:X_n\to X_{n-1}\quad\text{and}\quad s_i\:X_n\to X_{n+1},\quad\text{for }0\leq i\leq n, \]
induced by the morphisms $d^i$ and $s^i$ in the category $\bfDelta$,
are called the \term{face maps} and the \term{degeneracy maps}, respectively.

\begin{example}
    We construct some examples of simplicial sets.
    \begin{itms}
        \item The simplicial set $\Delta[n]$,
        as a simplicial complex, corresponds to the standard $n$-simplex.
        Its $k$-simplices are in 1--1 correspondence with 
        order-preserving maps $[k]\to[n]$,
        where $[n]$ can be seen as the set of vertices of $\Delta[n]$.
        \item Note that $\Delta[\bullet]$ is a cosimplicial object in $\cat{sSet}$.
        \item By removing the only non-degenerate $n$-simplex in $\Delta[n]$,
        we obtain its boundary $\partial\Delta[n]$.
        \item The simplicial set $S^n$ is defined to be $\Delta[n]\/\partial\Delta[n]$,
        where the quotient is done degreewise. \varqed
    \end{itms}
\end{example}

\begin{proposition}
    The category $\cat{sSet}$ admits all (small) colimits and limits, 
    which are defined degreewise, e.g.
    \[ (X\times Y)_n:=X_n\times Y_n. \]
\end{proposition}

\begin{proof}
    Exercise for the reader.
\end{proof}

For example, the product $\Delta[1]\times\Delta[1]$ is a solid square, which looks like
\[\begin{tikzcd}
    \bullet\ar[d]\ar[r]\ar[dr] & \bullet\ar[d] \\
    \bullet\ar[r] & \bullet\rlap{\ ,}
\end{tikzcd}\]
% \xymatrix{ \bullet\ar[d]\ar[r]\ar[dr] & \bullet\ar[d] \\ \bullet\ar[r] & \bullet\rlap{\ ,} }
with $2$ non-degenerate $2$-simplices and $5$ non-degenerate $1$-simplices,
but actually it has $3\times3=9=5+4$ possibly degenerate $1$-simplices in total.

\subsection{Geometric realisation}

A simplicial set is often pictured as
a simplicial complex together with an ordering of vertices for each simplex.
The idea of geometric realisation is that one can
forget the ordering and get a topological space.

\begin{proposition}
    Every simplicial set can be obtained from $\emptyset$
    by attaching $\Delta[n]$ along its boundary $\partial\Delta[n]$ and taking colimits. \qed
\end{proposition}

In the category $\cat{Top}$, we have the standard $n$-simplex $\Delta^n$.
If we put all these spaces together, we get a cosimplicial object $\Delta^\bullet$ in $\cat{Top}$,
which specifies ``how $\Delta[n]$ should look like in $\cat{Top}$''.
Given this data, we can easily define a functor
\[ |{\bullet}|\:\cat{sSet}\to\cat{Top} \]
by sending $\Delta[n]$ to $\Delta^n$,
and extending it to other simplicial sets by taking colimits.
This functor is called the \term{geometric realisation} functor.

Moreover, the functor $|{\bullet}|$ has a right adjoint called $\operatorname{Sing}$, defined by 
\[\operatorname{Sing}X:=\Hom_{\cat{Top}}(\Delta^\bullet,X),\]
which is exactly as we defined it before.

This construction can be generalised as follows.

\begin{construction}\label{con-3-a}
    Let $\cat C$ be a category with colimits.
    If we specify any cosimplicial object $\Delta^\bullet$ in $\cat{C}$,
    then we know ``how $\Delta[n]$ should look like in $\cat{C}$'',
    and we can similarly define an adjunction
    \[\begin{tikzcd}[column sep=5em]
        \cat{sSet}\ar[r,bend left=20,"\Delta{[n]}\mapsto\Delta^n"]\ar[r,phantom,"\bot"] &
        \rlap{$\cat{C}$\ .}\phantom{\cat{sSet}}\ar[l,bend left=20,"X\mapsto\Hom_{\cat C}(\Delta^\bullet\comma X)"]
    \end{tikzcd}\eqno◃\]
    % \xymatrix @C=5em {\cat{sSet}\ar@/^2ex/[r]^{\Delta{[n]}\mapsto\Delta^n}\ar@{}[r]|{\bot} & \rlap{\cat{C}\ .}\phantom{\cat{sSet}}\ar@/^2ex/[l]^{X\mapsto\Hom_{\cat C}(\Delta^\bullet,X)}}
\end{construction}

As we will see in the future,
many constructions related to simplicial sets are special cases of this construction.
Here is one example.

\begin{example}
    Let us take $\cat{C}=\cat{Ch}_R$, the category of cochain complexes of $R$-modules,
    where $R$ is a ring. Take
    \[\Delta^n:=\quad\Bigl(\cdots\to0\to
    \underset{\substack{\\[.5em](-n)}}{R^{\oplus\binom{n+1}{n+1}}}\to
    \cdots\to
    \underset{\substack{\\[.5em](-1)}}{R^{\oplus\binom{n+1}{2}}}\to
    \underset{\substack{\\[.5em](0)}}{R^{\oplus(n+1)}}\to0\to\cdots\Bigr)\]
    to be the simplicial chain complex of the standard $n$-simplex.
    This construction gives a functor $\cat{sSet}\to\cat{Ch}_R$,
    which computes the simplicial homology of a simplicial set.
    The composition
    \[\cat{Top}\xrightarrow{\operatorname{Sing}}\cat{sSet}\to\cat{Ch}_R\]
    computes the singular homology of a topological space. \varqed
\end{example}

Let us go back to the adjunction $|{\bullet}|\dashv\operatorname{Sing}$.
Actually, this is a Quillen equivalence between model categories.

\begin{definition}
    Let $0\leq i\leq n$.
    The simplicial set $\Lambda_i[n]$, called a \term{horn},
    is obtained from $\partial\Delta[n]$ by removing the face opposite to the $i$-th vertex.
\end{definition}

\begin{theorem}
    The category $\cat{sSet}$ has a \term{standard model structure}, with
    \begin{itms}
        \item $\cat W:=\{$weak homotopy equivalences of topological spaces$\}$.
        \item $\cat{Cof}:=\{$injections$\}$.
        \item $\cat{Fib}=\operatorname{RLP}\{\Lambda_i[n]\hookrightarrow\Delta[n]\mid0\leq i\leq n,\ n>0\}$.
        \item $\cat{Fib\cap W}=\operatorname{RLP}\{\partial\Delta[n]\hookrightarrow\Delta[n]\mid n\geq0\}$.
    \end{itms}
\end{theorem}

The proof is rather tedious and will not be presented here.
See~\cite[Theorem~3.6.5]{hovey}.

Finally, we state without proof the following result.

\begin{theorem}
    The adjunction $|{\bullet}|\dashv\operatorname{Sing}$
    is a Quillen equivalence between $\cat{sSet}$ and $\cat{Top}$.
\end{theorem}

See~\cite[Theorem 3.6.7]{hovey}.

The reader can show that the adjunction is a Quillen adjunction,
without using this theorem.

\subsection{Categories as simplicial sets}

Simplicial sets can be seen as a model for categories.
For example, the simplicial set $\Delta[2]$ can be seen as a diagram
\[\begin{tikzcd}[row sep=2em,column sep=1em]
    &\bullet\ar[dr] \\
    \bullet\ar[ur]\ar[rr]\ar[rr,phantom,bend left=45,"\Downarrow"] && \bullet\rlap{\ ,}
\end{tikzcd}\]
% \xymatrix @R=2em @C=1em{ &\bullet\ar[dr] \\ \bullet\ar[ur]\ar[rr]\ar@{}@/^2.5ex/[rr]|{\textstyle\Downarrow} && \bullet\rlap{\ ,} }
where the double arrow indicates that
the $2$-simplex ``witnesses'' the composition of the two arrows.
In an ordinary category, this diagram is just a chain of $2$ arrows
\[\bullet\to\bullet\to\bullet\ .\]
As another example, the simplicial set $\Delta[2]\times\Delta[1]$
corresponds to a diagram
\[\begin{tikzcd}
    \bullet\ar[d]\ar[r] & \bullet\ar[d]\ar[r] & \bullet\ar[d] \\
    \bullet\ar[r] & \bullet\ar[r] & \bullet\rlap{\ .}
\end{tikzcd}\]
% \xymatrix{ \bullet\ar[d]\ar[r] & \bullet\ar[d]\ar[r] & \bullet\ar[d] \\ \bullet\ar[r] & \bullet\ar[r] & \bullet\rlap{\ .} }

\begin{definition}
    Let $\cat C$ be a (small) category.
    The \term{nerve} of $\cat C$ is a simplicial set denoted by $\operatorname{N}(\cat C)$.
    Its $n$-simplices are chains of $n$ arrows
    \[\bullet\to\bullet\to\cdots\to\bullet\]
    in $\cat C$.
    Its $0$-th (resp.\ $n$-th) face map is defined by discarding the first arrow (resp.\ the last arrow).
    For $0<i<n$, its $i$-th face map is defined by composing its $i$-th and $(i+1)$-th maps.
    Its degeneracy maps are defined by inserting identity morphisms.
\end{definition}

The reader can verify that, in the above examples,
the nerve of the categories are the corresponding simplicial sets.

\begin{remark}
    In fact, this is another special case of (\ref{con-3-a}).
    Namely, the cosimplicial object $\Delta^\bullet$ in $\cat{Cat}$, given by 
    \[\Delta^n:=\quad\bullet\to\bullet\to\cdots\to\bullet\]
    with $n$ consecutive arrows, gives an adjunction
    \[\begin{tikzcd}
        \cat{sSet}\ar[r,bend left=30,"\Ho"]\ar[r,phantom,"\bot"] &
        \rlap{$\cat{Cat}$\ .}\phantom{\cat{sSet}}\ar[l,bend left=30,"\operatorname{N}"]
    \end{tikzcd}\]
    % \xymatrix{ \cat{sSet}\ar@/^/[r]^{\Ho}\ar@{}[r]|{\bot} & \cat{Cat}\rlap{\ .}\ar@/^/[l]^{\operatorname{N}} }
    The right adjoint is precisely the nerve functor defined above. \varqed
\end{remark}

Note that not all simplicial sets are categories.
For example, consider $\Lambda_1[2]$, which has two arrows that cannot be composed.
In fact, if we define the \term{spine} $\operatorname{Sp}(n)\subset\Delta[n]$
to consist of all vertices and the edges $[i,i+1]$ for $0\leq i<n$,
then a simplicial set $X$ is the nerve of a category,
if and only if it satisfies the lifting property
\[\begin{tikzcd}
    \operatorname{Sp}(n)\ar[d,hook]\ar[r] & X \\
    \Delta[n]\ar[ur,dashed,"\exists!"']
\end{tikzcd}\]
% \xymatrix{ \operatorname{Sp}(n)\ar[d]\ar[r] & X \\ \Delta[n]\ar@{-->}[ur] }
for all $n\geq 0$.

\subsection{Kan complexes and quasi-categories}

Kan complexes are simplicial sets that looks like topological spaces,
and will be used to model $\infty$-groupoids.

\begin{definition}
    A \term{Kan complex} is a fibrant simplicial set.
    In other words, it satisfies the \term{horn extension property},
    that is, the lifting property
    \[\begin{tikzcd}
        \Lambda_i[n]\ar[d,hook]\ar[r] & X \\
        \Delta[n]\ar[ur,dashed]
    \end{tikzcd}\]
    % \xymatrix{ \Lambda_i[n]\ar[d]\ar[r] & X \\ \Delta[n]\ar@{-->}[ur] }
    for all $0\leq i\leq n$, $n>0$.
\end{definition}


\begin{example}
    The simplicial set $\Delta[n]$ is \emph{not} a Kan complex if $n\geq1$.
    Namely, consider the horn
    \[\begin{aligned}
        \Lambda_0[2] &\to \Delta[n], \\
        0,1,2 &\mapsto 0,1,0.
    \end{aligned}\]
    Then it is impossible to extend this horn to make a $\Delta[2]$.
    The reason is that $\Delta[n]$, seen as a category,
    does not have inverses of morphisms. In other words, it does not look like a groupoid,
    while Kan complexes must look like groupoids. \varqed
\end{example}

\begin{example}
    For any topological space $X$, the simplicial set $\operatorname{Sing}X$
    is a Kan complex, as one can show easily.
    Therefore, for any simplicial set $S$, the simplicial set
    $\operatorname{Sing}|S|$
    can be used as a fibrant replacement of $S$. \varqed
\end{example}

As we have seen above, Kan complexes look like groupoids with higher structures.
In a Kan complex, $1$-morphisms are invertible,
but only up to a $2$-morphism, i.e.\ $2$-simplex.
The horn extension property ensures that all higher morphisms
are also invertible, up to even higher morphisms.
This is exactly what an $\infty$-groupoid should be.

\begin{definition}
    An \term{$\infty$-groupoid}, or an \term{$(\infty,0)$-category},
    is a Kan complex.
\end{definition}

For example, for a topological space $X$,
the Kan complex $\operatorname{Sing}X$ can be seen as 
the ``fundamental $\infty$-groupoid'' of $X$.

\begin{definition}
    A \term{homotopy type} is a homotopy type of $\infty$-groupoids,
    i.e.\ an element of $\Ho(\cat{sSet})$, which is equivalent to the category
    $\Ho(\cat{Top})\simeq\cat{hCW}$.
\end{definition}

Next, we wish to define $(\infty,1)$-categories in a similar way.
The following table shows that the extension of horns corresponds to 
properties of a category,
assuming that we are considering the nerve of an ordinary category.
\begin{center}
    \begin{tabular}{cc}
        \textbf{Horn} & \textbf{Property} \\ \hline
        $\Lambda_0[2]$ & every morphism is left invertible \\
        $\Lambda_1[2]$ & composition of morphisms \\
        $\Lambda_2[2]$ & every morphism is right invertible \\
        $\Lambda_0[3]$ & every morphism is an epimorphism \\
        $\Lambda_1[3]$, $\Lambda_2[3]$ & associativity of composition \\
        $\Lambda_3[3]$ & every morphism is a monomorphism \\
        otherwise & (satisfied by any category)
    \end{tabular}
\end{center}
As an exercise, the reader should verify everything in the table.

We notice that the extension of
$\Lambda_0[2]$, $\Lambda_2[2]$, $\Lambda_0[3]$ and $\Lambda_3[3]$
can not be satisfied by all categories,
while the extension of ``inner horns'' $\Lambda_i[n]$ for $0<i<n$
describes properties that any category should satisfy.

\begin{definition}
    A \term{quasi-category}, or an \term{$(\infty,1)$-category},
    is a simplicial set having the lifting property
    \[\begin{tikzcd}
        \Lambda_i[n]\ar[d,hook]\ar[r] & X \\
        \Delta[n]\ar[ur,dashed]
    \end{tikzcd}\]
    % \xymatrix{ \Lambda_i[n]\ar[d]\ar[r] & X \\ \Delta[n]\ar@{-->}[ur] }
    for $0<i<n$. In other words, \term{inner horns} can be extended.
\end{definition}

The terminology is that quasi-categories are 
one of the various models for $(\infty,1)$-categories.
For this reason, we will stick to the term ``quasi-categories''.
Some people also call quasi-categories ``weak Kan complexes''.

\subsection{Roadmap}

A number of models and tools for studying $(\infty,1)$-categories 
will be used in the sequel. 
Since we have finished most of the definitions,
it is a good point now to draw a roadmap
as a preview of what we will encounter.

For a monoidal category $\cat V$,
let $\cat{Cat}_{\cat V}$ denote the category of categories enriched over $\cat V$.
(We ignore the set-theoretic issues here.)
Let $\cat{Model}$ denote the category of model categories.
Let $\cat{Model}_{\cat V}$ denote the category of $\cat V$-enriched model categories,
which we have not defined yet.
Let $\cat{QsCat}$ denote the category of quasi-categories
(which should really be replaced by $\cat{sSet}$ in the following diagram).

We have a diagram of categories 
\[\begin{tikzcd}[row sep=1ex]
    \cat{Model} \ar[dd,"\text{localise}"']
    \ar[ddddr,start anchor={[xshift=-.5ex]west},bend right=90,"\Ho"'] &
    \cat{Model}_{\cat{sSet}} \ar[l,"\text{forget}"'] 
    \ar[dd,"\text{forget}"'] \\ 
%
    && \cat{Cat}_{\cat{sMod}_R} \ar[r,bend left=20,"\text{Dold--Kan}"]
    \ar[r,phantom,"\bot"]
    \ar[dl,bend left=20,start anchor={[xshift=-1.5ex,yshift=1ex]},
    end anchor={[yshift=1ex]},pos=.7,"\text{forget}" {rotate=25,inner sep=1pt}] &
    \rlap{$\cat{Cat}_{\cat{Ch}_R}$\scriptsize\enspace(dg categories)}
    \phantom{\cat{Cat}_{\cat{sMod}_R}} \ar[l,bend left=20] \\
%
    \cat{QsCat} \ar[r,bend left=20,"\mathfrak C"]
    \ar[r,phantom,"\bot\rlap{$\scriptstyle\simeq$}"] \ar[ddr,bend right,"h"'] &
    \cat{Cat}_{\cat{sSet}} \ar[l,bend left=20,"\mathfrak N"] \ar[dd,"h"']
    \ar[ur,bend left=20,start anchor={[xshift=1ex,yshift=1ex]},
    end anchor={[yshift=.5ex]},pos=.75,"\text{free}" rotate=25]
    \ar[ur,phantom,yshift=1ex,"\bot" rotate=25]
    \ar[dr,bend left=20,start anchor={[yshift=-1ex]},
    end anchor={[xshift=-1.5ex,yshift=-1ex]},pos=.45,"|\bullet|" {rotate=-22,inner sep=1pt}]
    \ar[dr,phantom,yshift=-1ex,"\bot\scriptstyle\simeq" rotate=-22] \\
    &&\cat{Cat}_{\cat{Top}} \ar[dl,bend left,"h"]
    \ar[ul,bend left=20,start anchor={[xshift=-1ex,yshift=-1ex]},
    end anchor={[xshift=.5ex,yshift=-1ex]},pos=.25,"\operatorname{Sing}" rotate=-22]\\
%
    &\cat{Cat}_{\cat{hCW}} \ar[dd,"\pi_0"] \\ \  \\
    &\cat{Cat}\rlap{\ .}
\end{tikzcd}\hspace{3em}\]
Some of the maps are easy to define, while others will be defined later.
This diagram is ``commutative'', in a sense which will be made precise later on.
Everything above $\cat{Cat}_{\cat{hCW}}$ can be seen as
models for $(\infty,1)$-categories.
Homotopy theory in these categories are called \term{homotopy coherent},
as opposed to \term{homotopy commutative}, 
which refers to commutative diagrams in $\cat{Cat}_{\cat{hCW}}$.

As we can see in the diagram, homological algebra, which is done in a dg (differential graded) category,
is related to $(\infty,1)$-category theory.
This relationship will be studied a few sections later.

