One might notice the resemblance between a handlebody and a CW complex.
In fact, they are related in the following way.

\begin{definition}
Let $(N,A)$ be a good, weak or non-weak, relative handlebody.
We define its \term{associated CW complex} $(X,A)$ as follows.
Assume for $N$ that the handles are attached one at a time.
\begin{itms}
\item Let $X_0:=A$ as a trivial relative CW complex over $A$,
and let $p_0\:N_0\to X_0$ be the identity map.
\item If $N_1=N_0\cup_{\Phi_1}h^\lambda$,
we define $X_1:=X_0\cup_{\phi_1}D^\lambda$,
where $\phi_1:=p_0\circ\Phi_1|_{\partial D^\lambda\times 0}$.
We define a continuous map $p_1\:N_1\to X_1$ as follows.
Let $p_1$ agree with $p_0$ on $N_0$,
and let it be the projection to the core $D^\lambda\times0$ of $h^\lambda$.
\item We then define $X_2$ and $p_2$, and so on. 
\end{itms}
Thus $X:=\limdct X_i$ is a relative CW complex over $A$.
\end{definition}

\begin{proposition}\label{prop:ass-cw}
The associated CW complex is homotopy equivalent to the original, weak or non-weak, relative handlebody.
\end{proposition}

\begin{proof}
The weak case needs an extra first step. Note that $N_i$ is homotopy equivalent to $N_i\setminus\partial N_i$,
since if we take a collar neighbourhood $\partial N_i\times[0,+\infty)$,
then both spaces retract to $N_i\setminus\partial N_i\times[0,1)$.

We claim that each $p_i$ is a homotopy equivalence.
This is shown by induction on $i$. Since the projection onto the core is a homotopy equivalence
between $h^\lambda$ and $D^\lambda$, by \cite[Lemma~2.1.3]{may},
$N_i:=N_{i-1}\cup_{\Phi_i}h^\lambda$ and $X_i:=X_{i-1}\cup_{\phi_i}D^\lambda$
are homotopy equivalent through the induced map $p_i$.

Finally, by \cite[Lemma~2.1.10]{may}, the colimit map $p\:N\to X$ is a homotopy equivalence.
\end{proof}

As a consequence,
the (co)homology groups of a handlebody 
can be computed by the cellular (co)chain complex of 
its associated CW complex.
We shall define an analogous (co)chain complex associated to a handlebody.

\begin{definition}
Let $M,S,T$ be oriented $m$-, $s$- and $t$-manifolds without boundary,
with $S,T$ compact and $s+t=m$.
Let $f\:S\to M$, $g\:T\to M$ be embeddings.
If $f(S)$ is transverse to $g(T)$, then we define the
\term{intersection number} of $f,g$ to be
\[\#(f,g):=\sum_{p\in f(S)\cap g(T)}\pm1,\]
where the sign is decided by whether the vector space isomorphism
$\T_pM\simeq\T_pf(S)\oplus\T_pg(T)$
preserves $(+1)$ or reverses $(-1)$ the orientation.
\end{definition}

\begin{proposition}
Under the above assumptions, if $f$ is isotopic to $f'$ and $g$ is isotopic to $g'$,
such that $f'(S)$ is transverse to $g'(T)$, then $\#(f,g)=\#(f',g')$.
\end{proposition}

\begin{proof}
By (\ref{cor:isotopy-ext}), we may extend the isotopy from $g$ to $g'$ to an isotopy of $M$.
Thus we may assume $g=g'$.
For a proof in this case, see \cite[p.\,108]{gp}. 
\end{proof}

Therefore, the intersection number is well-defined for isotopy classes of maps.
This allows us to define the intersection number for submanifolds
that are not necessarily transverse.

\begin{definition}
Let $M,S,T$ be oriented $m$-, $s$- and $t$-manifolds without boundary,
where $S,T$ are compact submanifolds of $M$ with $s+t=m$.
Their \term{intersection number} $\#(S,T)$
is defined to be any $\#(f,g)$, such that $f$ is isotopic to the embedding $S\hookrightarrow M$,
$g$ is isotopic to $T\hookrightarrow M$, and $f(S)$ is transverse to $g(T)$.
\end{definition}

For any given $M,S,T$, such $f,g$ always exist.
This is a consequence of (\ref{cor:transverse}). 

\begin{definition}
Let $N$ be a good, weak or non-weak, $n$-handlebody,
and let $G$ be an abelian group.
The \term{handle chain complex} $C_\bullet(N;G)$ of $N$ is defined as follows.
\begin{itms}
\item $C_\lambda(N;G):=\bigoplus_\alpha G\cdot[\subsup h\alpha\lambda]$
for $\lambda=0,\dotsc,n$, and $0$ otherwise.
Here $\subsup h\alpha\lambda$ runs through all $\lambda$-handles of $N$.
\item $\partial_\lambda\:C_\lambda(N;G)\to C_{\lambda-1}(N;G)$ is defined on basis elements by
\[\partial_\lambda[\subsup h\alpha\lambda]:=\sum\#\bigl(\text{cobelt}(\subsup h\alpha\lambda),\ \text{belt}(\subsup h\beta{\lambda-1})\bigr)\,[\subsup h\beta{\lambda-1}],\]
where the sum is taken over all $(\lambda-1)$-handles $\subsup h\beta{\lambda-1}$
attached before $\subsup h\alpha\lambda$.
\end{itms}
If $N$ is locally finite, then we define the \term{handle cochain complex}
$C^\bullet(N;G)$ as follows.
\begin{itms}
\item $C^\lambda(N;G):=C_\lambda(N;G)$.
\item $d^\lambda\:C^\lambda(N;G)\to C^{\lambda+1}(N;G)$ is defined on basis elements by
\[d^\lambda[\subsup h\alpha\lambda]:=\sum\#\bigl(\text{belt}(\subsup h\alpha\lambda),\ \text{cobelt}(\subsup h\beta{\lambda+1})\bigr)\,[\subsup h\beta{\lambda+1}],\]
where the sum is taken over all $(\lambda+1)$-handles $\subsup h\beta{\lambda+1}$
attached after $\subsup h\alpha\lambda$.
\end{itms}
The relative versions, 
namely $C_\bullet(N,A;G)$ and $C^\bullet(N,A;G)$, are defined similarly.
\end{definition}

We need to prove that these are indeed chain complexes.
We do this by showing that the homological version
is isomorphic to the cellular chain complex
of a CW complex.

\begin{proposition}
Let $X$ be the associated CW complex of $N$.
If $N$ is orientable or $2G=0$,
then $C_\bullet(N;G)$ is isomorphic to the
cellular chain complex $\subsup C\bullet{\mathrm{cell}}(X;G)$.
\end{proposition}

\begin{proof}
Note that under the map $p_i\:N_i\to X_i$,
the intersection number corresponds to the
sum of local degrees
of the attaching map on the inverse image of $0\in D^{\lambda-1}$.
In the orientable case the orientation of each $\partial N_i$
may be chosen to be compatible with the handles,
so that degrees are counted correctly;
otherwise they are only correct modulo $2$.
\end{proof}

\begin{proposition}\label{prop:handle-homology}
Suppose $N$ is orientable or $2G=0$. Then 
\begin{itms}
    \item The chain complex $C_\bullet(N;G)$ computes
    the singular homology $H_\bullet(N;G)$.
    \item The cochain complex $C^\bullet(N;G)$, if defined, computes
    the singular cohomology with compact support $\subsup H{\mathrm c}\bullet(N;G)$.
\end{itms}
\end{proposition}

\begin{proof}
The first statement is immediate; we prove the second one.

If $N$ is compact (that is, finite), 
then $C^\bullet(N;G)$ is precisely the dual of $C_\bullet(N;G)$,
i.e.\ obtained by applying $\Hom_G(-,G)$.
By \cite[Theorem~3.5]{hatcher},
the dual of the cellular complex computes the singular cohomology.
Thus $C^\bullet(N;G)$ also computes the singular cohomology,
which is equal to singular cohomology with compact support.

In the general case, suppose that the handles in $N$ are attached
one at a time. This does not affect the chain complexes.
We use the fact that
\[\subsup H{\mathrm c}\bullet(X;G)
\simeq\limdct_{\raisebox{2pt}[0pt]{\scriptsize$K$ compact}} H^\bullet(X,X\setminus K;G)\]
for any space $X$ \cite[below~3.33]{hatcher}.
By local finiteness, for any compact $K\subset N$,
there exists $i$ such that $K\subset(\text{interior of $N_i$ in $N$})$.
By excision we have $H^\bullet(N,N\setminus K;G)\simeq H^\bullet(N_i,N_i\setminus K;G)$. Thus
\[\subsup H{\mathrm c}\bullet(N;G)\simeq\limdct\subsup H{\mathrm c}\bullet(N_i;G).\]
But each $N_i$ is compact, hence
$\subsup H{\mathrm c}\bullet(N_i;G)\simeq H^\bullet(N_i;G)$.
Note also that $C^\bullet(N;G)\simeq\limdct C^\bullet(N_i;G)$,
in the category of cochain complexes.
It remains to show that the functor $H^\bullet$ (of cochain complexes) commutes with taking colimits,
which is a standard result in homological algebra.
\end{proof}

\begin{remark}
To compute the singular cohomology $H^\bullet(N;G)$,
one may directly dualise $C_\bullet(N;G)$
by applying $\Hom_G(-,G)$. \varqed
\end{remark}

\begin{remark}
This gives a beautiful geometrical interpretation
of Poincar\'e duality for a manifold $M$,
which states that $H_\bullet(M;G)$ and $\subsup H{\mathrm{c}}{\bullet}(M;G)$ are dual to each other,
provided that $M$ is orientable or $2G=0$.
Namely, for a finite handlebody $N$,
one defines in the obvious way the \term{dual handlebody} of $N$,
which is diffeomorphic to $N$,
and its $\lambda$-handles correspond to the $(n-\lambda)$-handles of $N$.
Then the handle chain complex of $N$ is isomorphic to 
the handle cochain complex of the dual of $N$.\varqed
\end{remark}
