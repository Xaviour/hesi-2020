In algebraic topology,
we have seen the following analogy between topology and homological algebra.

\begin{center}
    \begin{tabular}{cc}
        \textbf{Topology} & \textbf{Homological Algebra} \\\hline
        spaces & chain complexes over $R$ \\
        homotopies & chain homotopies \\
        homotopy equivalences & chain homotopy equivalences \\
        homotopy groups $\pi_n(X)\simeq[S^n,X]$ & homology groups $H^n(X)\simeq[\text{``}S^n\text{''},X]$ \\
        weak homotopy equivalences & quasi-isomorphisms \\
        CW approximation & projective/injective resolutions \\
        homotopy category $\cat{hCW}$ & derived category $\cat{D}(R)$ \\
        suspension and looping $\Sigma\dashv\Omega$ & shifting $[1]\dashv[-1]$ \\
        \dots\dots & \dots\dots
    \end{tabular}%
    \medskip

    {\leftskip=1.5em \rightskip=1.5em \small
    ($R$ is a commutative ring;
    $[\ ,\ ]$ denotes the set of homotopy classes of maps;
    and ``$S^n$'' denotes the chain complex
    whose only non-zero term is $R$ at its $n$-th place.)

    }
\end{center}

Homotopical algebra is a language that unifies these two theories.
It is a general theory that also applies to many other situations.

\subsection{What is an \texorpdfstring{$\infty$}{∞}-category?}

The language of $\infty$-categories is the modern language for homotopical algebra.
It encodes data describing the ``higher structures'' of a category
that can not be seen in ordinary category theory.

Roughly speaking, an $\infty$-category consists of a class of objects,
a class of morphisms between them,
and moreover, there are higher morphisms between these morphisms.
Namely, there are $2$-morphisms between ordinary morphisms,
which can be seen as ``homotopies'' between morphisms.
There are $3$-morphisms between $2$-morphisms,
which can be seen as ``homotopies between homotopies'', and so on.

We will not give the definition for an $\infty$-category
until a few sections later, since defining them will require some preliminary work.
However, the following examples will give us 
a first impression of what an $\infty$-category looks like.

\begin{example} \label{eg-1-top}
    Consider the category $\cat{Top}$ of topological spaces.
    Seen as an $\infty$-category, it will consist of the following data.
    \begin{samepage}
        \begin{itms}
            \item Objects: topological spaces.
            \item $1$-Morphisms: continuous maps between topological spaces.
            \item $2$-Morphisms (between $1$-morphisms): homotopies between two maps
            with the same source and target.
            \item $3$-Morphisms (between $2$-morphisms): homotopies between homotopies of maps. \nopagebreak
            \item \dots\dots \varqed
        \end{itms}
    \end{samepage}
\end{example}

\begin{example}
    For a commutative ring $R$, consider the category $\cat{Ch}_R$ of cochain complexes over $R$.
    As an $\infty$-category, it will consist of the following data.
    \begin{itms}
        \item Objects: cochain complexes over $R$.
        \item $1$-Morphisms: chain maps between cochain complexes.
        \item $2$-Morphisms: chain homotopies between chain maps.
        \item $3$-Morphisms: chain homotopies between chain homotopies.
        \item \dots\dots \varqed
    \end{itms}
\end{example}

\begin{example}
    For a topological space $X$, consider the fundamental groupoid of $X$, denoted by $\cat{Π}(X)$.
    As an $\infty$-category, it will consist of the following data.
    \begin{itms}
        \item Objects: points of $X$.
        \item $1$-Morphisms: paths connecting two points.
        \item $2$-Morphisms: homotopies between paths, fixing endpoints.
        \item $3$-Morphisms: homotopies between homotopies.
        \item \dots\dots
    \end{itms}
    Note that the associativity law $f\circ(g\circ h)=(f\circ g)\circ h$
    does not hold in this example; it only holds ``up to homotopy''.
    This is one of the difficulties we will encounter in studying $\infty$-categories.
    \varqed
\end{example}

\subsection{Homotopy categories}

In many cases, $\infty$-categories arise as homotopy categories of ordinary categories with certain extra data.
We will now demonstrate this procedure through a concrete example.

\begin{definition}
    The category $\cat{hTop}$ consists of
    \begin{itms}
        \item Objects: topological spaces.
        \item Morphisms: homotopy classes of maps.
    \end{itms}
\end{definition}

A key observation is that $\cat{hTop}$ is obtained from $\cat{Top}$
by ``inverting the homotopy equivalences''. Let us make this precise.

\begin{definition}
    A \term{category with weak equivalences}
    is a pair $(\cat{C},\cat{W})$,
    where $\cat{C}$ is a category,
    and $\cat{W}\subset\operatorname{Mor}(\cat C)$ is a class of morphisms, such that
    \begin{itms}
        \item All isomorphisms of $\cat{C}$ are in $\cat{W}$.
        \item $\cat{W}$ satisfies \term{two-out-of-three}: for any diagram
        \[ \begin{tikzcd}[column sep=10pt,row sep=15pt]
            & \bullet\ar[rd,"g"] &   \\
            \bullet\ar[rr,"g\circ f"']\ar[ru,"f"] &   & \bullet 
        \end{tikzcd} \]
        in $\cat{C}$, if two of the arrows $f,g,g\circ f$ are in $\cat{W}$, then so is the third.
    \end{itms}
\end{definition}

For example, the pair $(\cat{Top},\cat{HoEq})$ is a category with weak equivalences,
where $\cat{HoEq}$ is the class of homotopy equivalences in $\cat{Top}$.

\begin{definition}
    Let $(\cat C,\cat W)$ be a category with weak equivalences.
    The \term{localisation} of $\cat C$ with respect to $\cat W$
    is a category $\cat C[\cat W^{-1}]$,
    together with a functor $\cat C\to\cat C[\cat W^{-1}]$,
    with the following universal property:
    \begin{itms}
        \item For any functor $F\:\cat C\to\cat D$ sending $\cat W$ to isomorphisms,
        there is a unique functor $\tilde F\:\cat C[\cat W^{-1}]\to\cat D$ up to a natural isomorphism,
        such that the diagram
        \[ \begin{tikzcd}[column sep=10pt]
            \cat C\ar[rr,"F"]\ar[dr] & & \cat D\\
            & \cat C[\cat W^{-1}]\ar[ur,dashed,"\exists!\ \tilde F"']
        \end{tikzcd} \]
        commutes up to a natural isomorphism.
    \end{itms}
\end{definition}

Roughly speaking, the category $\cat C[\cat W^{-1}]$
is obtained from $\cat C$ by making all the arrows in $\cat W$ invertible.
In fact, this idea can be formulated into an explicit construction
of the localisation $\cat C[\cat W^{-1}]$.

\begin{construction} \label{constr-1-locn}
    Let $(\cat C,\cat W)$ be a category with weak equivalences.
    Define $\cat C[\cat W^{-1}]$ to be the category with the same objects as $\cat C$,
    with $\Hom_{\cat C[\cat W^{-1}]}(X,Y)$
    the set of all possible sequences
    \[ X\to Z_1\gets Z_2\to\cdots \gets Z_n\to Y \]
    in $\cat C$,
    where all arrows going leftward are in $\cat W$,
    % 'quotiented' isn't a good way to say it, but it's standard terminology and I can't find a better expression.
    quotiented by the following relations:
    identity arrows can be dropped;
    adjacent arrows pointing to the same direction can be composed;
    adjacent arrows pointing to different directions can also be dropped
    if they represent the same morphism.
    It is then almost obvious that
    our construction does give a localisation with the desired universal property.

    The only problem is that $\Hom_{\cat C[\cat W^{-1}]}(X,Y)$
    may be too large to be a set; 
    however, we do not care about this problem for now, 
    and it is easily overcome by switching to a larger universe. \varqed
\end{construction}

\begin{proposition} \label{thm-1-htop}
    $\cat{hTop}\simeq\cat{Top}[\cat{HoEq}^{-1}]$.
\end{proposition}

\begin{proof}
    The natural functor $\cat{Top}\to\cat{hTop}$ sends the class $\cat{HoEq}$ to isomorphisms.
    Therefore, it induces a functor 
    \[ \cat{Top}[\cat{HoEq}^{-1}]\to\cat{hTop}, \]
    which is clearly full and essentially surjective.
    To prove that it is faithful, suppose $f,g \: X \to Y$
    are morphisms in $\cat{Top}$ which are sent to the same morphism in $\cat{hTop}$.
    Then $f$ and $g$ are homotopic. Let $H$ be the homotopy.
    Then in $\cat{Top}[\cat{HoEq}^{-1}]$, we have
    \[ \begin{aligned}
        f \quad &= \quad X \xrightarrow{i_0} X \times I \xrightarrow{H} Y \\
        &= \quad X \xrightarrow{i_0} X \times I
        \xrightarrow{\operatorname{pr}_1} X
        \xleftarrow{\operatorname{pr}_1} X \times I \xrightarrow{H} Y \\
        &= \quad X \xrightarrow{\mathbb{1}_X} X
        \xleftarrow{\operatorname{pr}_1} X \times I \xrightarrow{H} Y \\
        &= \quad X \xleftarrow{\operatorname{pr}_1} X \times I \xrightarrow{H} Y \\
        &= \quad g,
    \end{aligned} \]
    where $\operatorname{pr}_1$ denotes projection onto the first factor,
    and $i_0$ denotes the obvious inclusion.
\end{proof}

In fact, we will see that localisation gives rise to higher structure.
In this example, the ordinary category $\cat{hTop}$
is just the first layer of information that we get from localisation.
The full information is retained in an $\infty$-category,
which is, in this case, the $\infty$-category $\cat{Top}$ 
given in (\ref{eg-1-top}).

In this case,
the higher structure has a very clear description
by homotopies and higher homotopies.
One might expect that in similar scenarios,
such as in homological algebra,
when we try to invert the quasi-isomorphisms,
the resulting $\infty$-category can also be described 
in terms of homotopies and higher homotopies.
In fact, this will be one of our main goals,
and will motivate the notion of a \emph{model category}.

Model categories are categories with weak equivalences,
together with some extra structures
that will help us substantially in computations related to
$\infty$-categories obtained by localisation.
Here is our mind-map.
\[ \begin{tikzcd}[column sep=3em,row sep=2em]
    \overset{\text{model category}}{(\cat C,\cat W,\cat{Cof},\cat{Fib})} \ar[dr,"\text{present }(\text{easier})"] \\
    (\cat C,\cat W) \ar[u,dashed,"\text{find a structure}"] \ar[r,"\text{localise}","(\text{hard to describe})"']
    & \infty\text{-category }\cat C[\cat W^{-1}] \ar[d,"\text{forget}"] \\
    & \text{ordinary category }\cat C[\cat W^{-1}]
\end{tikzcd} \]

\subsection{A naive attempt on higher categories}

We will now try to give a simple, but ``wrong'',
definition of higher categories.
Keeping in mind that higher categories
are just categories with higher dimensional arrows,
we will formulate this idea into a rigorous definition.

\begin{definition}
    A \term{monoidal category} is a category $\cat C$, together with
    \begin{itms}
        \item An object $1\in\cat C$, called the \term{unit}.
        \item A functor $\otimes\:\cat C\times\cat C\to\cat C$,
    \end{itms}
    such that there are natural isomorphisms
    \[\setlength{\arraycolsep}{.2em} \begin{array}{rrl}
        a\:&(X\otimes Y)\otimes Z&\simto X\otimes(Y\otimes Z),\\[3pt]
        l\:&1\otimes X&\simto X,\\[3pt]
        r\:&X\otimes 1&\simto X,
    \end{array} \]
    so that the diagrams
    \[ \begin{tikzcd}[column sep=-5em]
        & & \hspace*{2em}(X\otimes Y)\otimes(Z\otimes W)\hspace*{2em}\ar[rrd] \\
        ((X\otimes Y)\otimes Z)\otimes W\ar[rru]\ar[dr] &&&& X\otimes(Y\otimes(Z\otimes W))\\
        & (X\otimes (Y\otimes Z))\otimes W\ar[rr] && X\otimes((Y\otimes Z)\otimes W)\rlap{\ ,}\ar[ur]
    \end{tikzcd} \]
    \[ \begin{tikzcd}[column sep=0]
        (X\otimes 1)\otimes Y\ar[dr]\ar[rr]&&X\otimes(1\otimes Y)\ar[dl]\\
        &X\otimes Y\rlap{\ ,}
    \end{tikzcd}
    \qquad
    \begin{tikzcd}
        1\otimes 1\ar[d,bend left,"r"]\ar[d,bend right,"l"'] \\ 1
    \end{tikzcd}
    \]
    are commutative for any $X,Y,Z,W\in\cat C$.
\end{definition}

The first diagram is also called the \emph{pentagon axiom}.
One can prove that even if we have more than $4$ objects,
the pentagon axiom ensures that the ``associahedron'' diagrams are commutative.

For example, the following triples $(\cat C,\otimes,1)$
are all examples of monoidal categories:
\begin{itms}
    \item $(\cat{Set},\times,*)$, where $*$ denotes the singleton set.
    \item $(\cat{Set},\sqcup,\emptyset)$, where $\sqcup$ denotes disjoint union.
    \item $(\text{any category with products},\times,*)$, where $*$ denotes the terminal object, i.e.\ the empty product.
    \item $(\text{any category with coproducts},\sqcup,\emptyset)$, where $\emptyset$ denotes the initial object, i.e.\ the empty coproduct.
    \item $(\cat{Ch}_R,\otimes,R)$,
    where the unit $R$ is concentrated in degree $0$, and the tensor product is given by
    \[ (C\otimes D)^n\coloneq\bigoplus_{p+q=n}C^p\otimes_R D^q. \]
\end{itms}

\begin{definition}
    Let $\cat V$ be a monoidal category.
    A \term{$\cat V$-enriched category} $\cat C$
    consists of the following data.
    \begin{itms}
        \item A class of objects.
        \item For any $X,Y\in\cat C$, a hom-object $\Hom_{\cat C}(X,Y)\in\cat V$.
        \item For any $X,Y,Z\in\cat C$, a composition map
        \[\circ\ \:\Hom_{\cat C}(Y,Z)\otimes\Hom_{\cat C}(X,Y)\to\Hom_{\cat C}(X,Z)\]
        in $\cat V$.
        \item For any $X\in\cat C$, an identity morphism \[\mathbb1_X:1\to\Hom_{\cat C}(X,X)\]
        in $\cat V$.
        We think of $\mathbb1_X$ as an ``element'' of $\Hom_{\cat C}(X,X)$,
        but since this object is not a set,
        we consider morphisms $1\to\Hom_{\cat C}(X,X)$
        as its ``elements''.
    \end{itms}
    They satisfy the following conditions.
    \begin{itms}
        \item Composition is associative.
        \item Composing with the identity morphism gives the original morphism.
    \end{itms}
    We leave it to the reader to formulate these two axioms rigorously.
\end{definition}

For example,
an ordinary category is a category
enriched over $(\cat{Set},\times,*)$.

When we have a forgetful functor $\cat V\to\cat{Set}$
preserving the monoidal structure,
we may think of a $\cat V$-enriched category
as a category with extra structures,
as in the following examples.

\begin{example}
    The category $\cat{Top}$ is enriched over itself,
    since for $X,Y\in\cat{Top}$,
    the set $\Hom_{\cat{Top}}(X,Y)$ can be given a natural topology,
    namely the compact open topology,
    so that composition is continuous. \varqed
\end{example}

\begin{example}\label{eg-1-c}
    The category $\cat{Ch}_R$ is enriched over itself.
    For $X,Y\in\cat{Ch}_R$, we define a cochain complex $\sHom(X,Y)$ by
    \[ \sHom(X,Y)^n:=\{f=\{f^k\:X^k\to Y^{k+n}\}_{k\in\mathbb Z}\}, \]
    where $f$ is not necessarily a chain map,
    and the differential is given by
    \[ df:=d\circ f-(-1)^{\deg f}f\circ d. \]
    The chain maps are the $0$-cocycles of this cochain complex.
    Such a cocycle corresponds to a chain map
    from the unit $1\in\cat{Ch}_R$ to this cochain complex, which is,
    in a sense, an ``element'' of this cochain complex. \varqed
\end{example}

\begin{example}
    The category $\cat{Cat}$ of all (small) categories
    is enriched over itself.
    This is because for any two categories $X,Y$,
    we may form their functor category $\operatorname{Fun}(X,Y)$,
    whose objects are functors $X\to Y$,
    and morphisms are natural transformations between these functors. \varqed
\end{example}

Using the language of enriched categories,
we can now give a first definition of higher categories.

\begin{definition}
    A \term{strict $2$-category} is a category enriched over $\cat{Cat}$.
\end{definition}

For example, $\cat{Cat}$ itself is a strict $2$-category.
We may think of functors as its $1$-morphisms,
and natural transformations as its $2$-morphisms.

Any $\cat{Top}$-enriched category can be regarded as a
strict $2$-category,
since we may replace its hom-spaces by
their fundamental groupoids.
In this case, the $2$-morphisms are just
paths connecting the $1$-morphisms in the hom-space.
For example, $\cat{Top}$ itself is a strict $2$-category.
Its $2$-morphisms are homotopies between maps.

However, for a topological space $X$,
the fundamental groupoid $\cat{Π}(X)$ is NOT a strict $2$-category.
This is because, as we have mentioned,
the composition in $\cat{Π}(X)$ is not strictly associative.
This indicates that our definition is too strict,
and is somehow ``wrong''.

\begin{definition}
    Inductively, we define a \term{strict $n$-category} as
    a category enriched over the category of $(n-1)$-categories.
\end{definition}

This gives rise to a definition of an $\infty$-category,
which is too strict and will not be used in the future.

\begin{definition}
    A \term{strict $\omega$-category} is a sequence
    \[ \cat C_1\hookrightarrow\cat C_2\hookrightarrow\cat C_3\hookrightarrow\cdots, \]
    where $\cat C_n$ is a strict $n$-category,
    such that $\cat C_{n-1}$ is obtained from $\cat C_n$
    by discarding all the $n$-arrows.
\end{definition}

\subsection{\texorpdfstring{$(n,r)$}{(n, r)}-categories}

Although we have not defined $n$-categories and $\infty$-categories in general,
we now have some intuitive ideas about what they are,
and we can talk about them in a semi-rigorous way.

\begin{definition}
    Let $0\leq r\leq n$ and $n>0$.
    An \term{$(n,r)$-category}
    is an $n$-category in which
    all $r+1,r+2,\dotsc,n$-morphisms are invertible.
\end{definition}

Since we are not only talking about strict $n$-categories,
``invertible'' actually means ``invertible up to a higher homotopy'',
or in other words, ``having a homotopy inverse''.
Let us look at some examples.

\begin{itms}
    \item An ordinary category is a $(1,1)$-category.
    \item An ordinary groupoid is a $(1,0)$-category.
    \item The strict $n$-categories defined above are $(n,n)$-categories.
    A strict $\omega$-category is an $(\infty,\infty)$-category.
    \item $(\infty,0)$-categories are called $\infty$-groupoids.
    They are very important objects, since they correspond to homotopy types.
    In fact, there is a theory called Homotopy Type Theory \cite{hott},
    which aims to rebuild the foundations of mathematics,
    using $\infty$-groupoids instead of sets as the basic building blocks.
    \item A category enriched over the category of $(n,r)$-categories
    is an $(n+1,r+1)$-category.
    \item $\cat{Top}$ is an $(\infty,1)$-category,
    since every homotopy is invertible.
    Namely, a homotopy composed with its own inverse
    is homotopic to the identity homotopy.
    \item $\cat{Ch}_R$ is an $(\infty,1)$-category,
    since composition of homotopies is addition of
    maps between cochain complexes,
    and for a chain homotopy,
    adding its additive inverse gives the zero map,
    which corresponds to the identity homotopy.
    \item For a topological space $X$,
    the fundamental groupoid $\cat{Π}(X)$ is
    an example of an $\infty$-groupoid.
    It describes the homotopy type of $X$.
\end{itms}

Regarding the fifth point,
we may extend the notion of $(n,r)$-categories to the case $n<1$.
In fact, we will see in the future
that the correct notions are as follows.
\begin{itms}
    \item $(0,0)$-categories are sets.
    \item $(0,1)$-categories are partially ordered sets.
    \item $(-1,0)$-categories are either $\emptyset$ or singleton sets. In other words, they are truth values.
    \item $(-2,0)$-categories are singleton sets.
\end{itms}

Higher category theorists believe that
the bottom layer should be $-2$,
and they think that it is more natural to renumber everything
so that $-2$ becomes $0$.
Thus, in the skyscraper of mathematics,
logic lives on the 1st floor;
set theory lives on the 2nd floor;
and category theory lives on the 3rd floor.

From this viewpoint,
it seems more natural to consider all the floors as a whole.
That is possibly why homotopy type theorists wish to replace set theory by
higher category theory as the foundation of mathematics.
