In the introduction, we mentioned that
if $(\cat{C},\cat{W})$ is a category with weak equivalences,
then we get an $\infty$-category $\cat{C}[\cat{W}^{-1}]$
by inverting the arrows in $\cat{W}$.
If moreover, the pair $(\cat{C},\cat{W})$ comes from a model structure,
which is usually the case,
then this $\infty$-category can be described more easily.

In this section, we aim to reformulate this idea rigorously,
and we will see in the following sections
how the model structure is going to help us.

\subsection{Localising an infinity category}

First of all, we describe the process of localising
a category with weak equivalences $(\cat{C},\cat{W})$
to get an $\infty$-category $\cat{C}[\cat{W}^{-1}]$.
We generalise the situation by also allowing $\cat{C}$ to be an $\infty$-category.

\begin{definition}
    Let $\cat{C}$ be a quasi-category,
    and let $\cat{W}$ be a set of edges (i.e., $1$-simplices) of $\cat{C}$.
    The \term{localisation} $\cat{C}[\cat{W}^{-1}]$, if it exists,
    is a quasi-category defined by the following universal property:
    \begin{itms}
        \item For any quasi-category $\cat{D}$, and any diagram without the dashed arrow 
        \[ \begin{tikzcd} [column sep=1em]
            \cat{C} \ar[rr,"F"] \ar[dr] && \cat{D} \\
            & \cat{C}[\cat{W}^{-1}] \rlap{\ ,} \ar[ur,dashed,"\exists!"']
        \end{tikzcd} \]
        such that $F$ sends $\cat{W}$ to invertible edges in $\cat{D}$,
        there exists a ``unique'' dashed arrow making the diagram commute.
    \end{itms}
    The ``uniqueness'' should be formulated in an $\infty$-categorical way:
    we require that
    \[ \operatorname{Fun}(\cat{C}[\cat{W}^{-1}], \cat{D}) \simto
        \operatorname{Fun}^{\cat{W}\mapsto\mathrm{eq}}(\cat{C},\cat{D}) \]
    is an equivalence of quasi-categories,
    where the right hand side denotes the full subcategory of $\operatorname{Fun}(\cat{C},\cat{D})$
    spanned by those functors sending $\cat{W}$ to invertible edges.
\end{definition}

As with all things determined by universal properties,
the localisation is unique up to an equivalence, if it exists.

\begin{remark}
    In fact, the localisation is unique up to an equivalence,
    which is unique up to a higher equivalence,
    which is unique up to an even higher equivalence, and so on. 
    In ordinary category theory, we often hear people say that 
    an object is ``unique up to a unique isomorphism''.
    This is a low dimensional truncation of our notion of uniqueness,
    because no (non-trivial) higher equivalences exist in an ordinary category. \varqed
\end{remark}

To prove the existence of such a localisation,
we need to recall the notion of marked simplicial sets.

\begin{definition}
    A \term{marked simplicial set} is a pair $(X,E)$,
    where $X$ is a simplicial set,
    and $E$ is a set of edges of $X$ which contains all degenerate edges.
    The edges in $E$ are called the \term{marked edges}.
\end{definition}

Recall that a map of marked simplicial sets from $(X,E)$ to $(Y,F)$
is a map of simplicial sets $f\:X\to Y$, such that $f(E)\subset F$.
As we have mentioned in the previous section,
the category $\cat{sSet}^+$ of all marked simplicial sets 
carries a model structure, which we shall describe now.

\begin{definition}
    Let $X$ be a simplicial set.
    \begin{itms}
        \item The marked simplicial set $X^\sharp$
        is the marked version of $X$ in which all edges are marked.
        \item The marked simplicial set $X^\flat$
        is the marked version of $X$ in which only the degenerate edges are marked.
        \item If $X$ is a quasi-category, then the marked simplicial set $X^\natural$
        is the marked version of $X$ in which the invertible edges are marked.
    \end{itms}
\end{definition}

\begin{theorem}
    The category $\cat{sSet}^+$ has the \term{(co)cartesian model structure}, in which
    \begin{itms}
        \item The cofibrations are the injections.
        \item The weak equivalences are the maps $f\:X\to Y$
        such that for any quasi-category $Z$, we have an equivalence of quasi-categories
        \[ \Hom_{\cat{sSet}^+}(Y,Z^\natural) \simto \Hom_{\cat{sSet}^+}(X,Z^\natural), \]
        where both sides are full subcategories of the mapping spaces of simplicial sets.
        \item The fibrant objects are those which are isomorphic to $X^\natural$
        for some quasi-category $X$.
    \end{itms}
\end{theorem}

See \cite[Proposition~3.1.3.7]{htt}.

The name of the model structure may seem a little strange.
In fact, it is a special case of the two model structures on $\cat{sSet}^+_{\smash{\/}S}$
when $S=\{*\}$. These two model structures were used in the formulation 
of Grothendieck construction for (co)cartesian fibrations in the last section.

The existence of localisations follows as an immediate consequence.

\begin{corollary}
    Let $\cat{C}$ be a quasi-category,
    and let $\cat{W}$ be a set of edges of $\cat{C}$.
    Then $\cat{C}[\cat{W}^{-1}]$ exists.
\end{corollary}

\begin{proof}
    Without loss of generality, we may assume that $\cat{W}$ contains all degenerate edges.
    Then $(\cat{C},\cat{W})$ is a marked simplicial set.
    Let $\cat{C}[\cat{W}^{-1}]:=R(\cat{C},\cat{W})$ be its fibrant replacement,
    which is a fibrant object, and thus a quasi-category.
    Then for any quasi-category $\cat{D}$, we have 
    \[ \Hom_{\cat{sSet}^+}\bigl(\cat{C}[\cat{W}^{-1}],\ \cat{D}^\natural\bigr) \simto
       \Hom_{\cat{sSet}^+}\bigl((\cat{C},\cat{W}),\ \cat{D}^\natural\bigr), \]
    as desired.
\end{proof}

This proof is terribly abstract, and it does not give us a clue on 
how to work with it in concrete examples.
However, there is a construction given by Dwyer and Kan \cite{dk-calc} which is 
much more explicit.

\begin{construction}
    Let $(\cat{C},\cat{W})$ be a category with weak equivalences,
    in the ordinary sense.
    The simplicial category $\cat{L}(\cat{C},\cat{W})$ is constructed as follows.
    It has the same objects as $\cat{C}$, and for $X,Y\in\cat{C}$,
    the simplicial set $\Hom_{\cat{L}(\cat{C},\cat{W})}(X,Y)$ is defined by 
    \[ \Hom_{\cat{L}(\cat{C},\cat{W})}(X,Y)_n := \left\{
        \begin{tikzcd}[row sep={1.2em,between origins},column sep=2em]
            & \bullet\ar[dd] & \bullet\ar[l]\ar[r]\ar[dd] & \cdots & \bullet\ar[l]\ar[dd]\ar[dddr] \\\\
            & \bullet\ar[dd] & \bullet\ar[l]\ar[r]\ar[dd] & \cdots & \bullet\ar[l]\ar[dd]\ar[dr] \\
            X \ar[uuur]\ar[ur]\ar[dddr] &&&&& Y \\
            & \vdots\ar[dd] & \vdots\ar[dd] && \vdots\ar[dd] \\\\
            & \bullet & \bullet\ar[l]\ar[r] & \cdots & \bullet\ar[l]\ar[uuur]
        \end{tikzcd}
    \right\}\raisebox{-3em}{$\Bigg/$}\mkern-6mu\raisebox{-3.5em}{$\sim$}, \]
    where 
    \begin{itms}
        \item There are $n+1$ rows and any number of columns.
        Each dot in the diagram (except in the ellipses) represents an object of $\cat{C}$.
        \item The horizontal arrows may go to the left or the right,
        but the arrows in the same column must go in the same direction.
        \item The arrows going downwards or leftwards are in $\cat{W}$.
        \item The equivalence relation $\sim$ is generated by 
        composition of adjacent horizontal arrows in the same direction.
    \end{itms}
    The simplicial category $\cat{L}(\cat{C},\cat{W})$ is called the 
    \term{hammock localisation} of $(\cat{C},\cat{W})$,
    since the above diagram looks like a hammock. \varqed
\end{construction}

By this construction, we easily see that the homotopy category of $\cat{L}(\cat{C},\cat{W})$
is the ordinary category which we denoted by $\cat{C}[\cat{W}^{-1}]$ in (\ref{constr-1-locn}).

\begin{theorem}
    Let $(\cat{C},\cat{W})$ be an ordinary category with weak equivalences.
    Then there is a weak equivalence of marked simplicial sets 
    \[ (\cat{C},\cat{W}) \simeq (\mathfrak{N}R\cat{L}(\cat{C},\cat{W}))^\natural, \]
    where $R$ denotes the fibrant replacement functor
    from $\cat{Cat}_{\cat{sSet}}$ to $\cat{Cat}_{\cat{Kan}}$.
    In particular, there is an equivalence of quasi-categories
    \[ \cat{C}[\cat{W}^{-1}] \simeq \mathfrak{N}R\cat{L}(\cat{C},\cat{W}). \]
\end{theorem}

See \cite[\S1.2]{hinich13}.

As a corollary, the homotopy category of $\cat{C}[\cat{W}^{-1}]$
is equivalent to the category defined in (\ref{constr-1-locn}).

\subsection{Simplicial model categories}

If a model category has a compatible simplicial structure,
then the $\infty$-category obtained by localisation
will be much easier to work with.
This type of model categories are called \emph{simplicial model categories}.

Defining the compatibility conditions between the model structure
and the simplicial structure will require some
preliminaries in monoidal category theory and enriched category theory.

For simplicity, we will only be concerned with
monoidal categories that are \emph{symmetric},
i.e., the tensor product is commutative,
which is almost always the case.

\begin{definition}
    A \term{symmetric monoidal category} consists of
    \begin{itms}
        \item A monoidal category $\cat{V}$.
        \item A natural equivalence
        \[ T_{X,Y} \: X \otimes Y \to Y \otimes X \]
        for $X,Y \in \cat{V}$,
    \end{itms}
    such that
    \begin{itms}
        \item $T \circ T = \mathbb{1}$.
        \item For any $X,Y,Z \in \cat{V}$, the diagram
        \[ \begin{tikzcd}
            (X\otimes Y)\otimes Z\ar[r]\ar[d] &
            (Y\otimes X)\otimes Z\ar[r] &
            Y\otimes(X\otimes Z)\ar[d] \\
            X\otimes(Y\otimes Z)\ar[r] &
            (Y\otimes Z)\otimes X\ar[r] &
            Y\otimes(Z\otimes X)
        \end{tikzcd} \]
        commutes.
    \end{itms}
\end{definition}

It can be shown that the last axiom above is enough
to ensure that all possible ways to form the tensor product 
of finitely many objects will give a unique result up to 
a unique isomorphism.
The essential reason is that $\cat{Cat}$ is a $2$-category 
and has no higher structure more than the level $2$.
This will be explained in later sections.

For example, all the examples of monoidal categories
that we have mentioned above are symmetric.

\begin{example}
    Let us give an example of a non-symmetric monoidal category.
    Consider a non-commutative ring $R$,
    and consider the category of bimodules over $R$,
    with its monoidal structure given by the tensor product of bimodules.
    Then this category is a non-symmetric monoidal category.
    \varqed
\end{example}

\begin{definition}
    A symmetric monoidal category $\cat{V}$ is said to be \term{closed},
    if there exists an \term{internal hom} functor 
    \[ (-)^{(-)} \: \cat{V} \times \cat{V}\op \to \cat{V}, \]
    and a natural isomorphism 
    \[ \Hom_{\cat{V}}(X \otimes Y,\ \cat{Z}) \simeq 
    \Hom_{\cat{V}}(X,\ Z^Y) \]
    for $X,Y,Z\in\cat{V}$.
    In short, the tensor product has a right adjoint.
\end{definition}

To avoid awkward notation,
we sometimes write $\Hom_{\cat{V}}(X,Y)$ for $Y^X$.

\begin{example}
    Here are some examples and non-examples of closed symmetric monoidal categories.
    \begin{itms}
        \item The symmetric monoidal category $(\cat{Set},\times)$ is closed,
        with internal hom given by the powering of sets.

        \item The symmetric monoidal category $(\cat{sSet},\times)$ is closed,
        with internal hom given by the mapping space of simplicial sets (\ref{def-4-map}).

        \item The symmetric monoidal category $(\cat{Top},\times)$ is \emph{not} closed.
        This is why algebraic topologists like to use the category of
        \emph{compactly generated spaces}, which is closed with the $\times$ monoidal structure,
        with internal hom given by the mapping space with compact open topology. \varqed
    \end{itms}
\end{example}

\begin{definition}
    Let $\cat{V}$ be a closed symmetric monoidal category,
    and let $\cat{C}$ be a category enriched over $\cat{V}$.
    \begin{itms}
        \item $\cat{C}$ is \term{tensored} over $\cat{V}$,
        if there exists a functor
        \[ {-}\otimes{-} \: \cat{V}\times\cat{C} \to \cat{C}, \]
        with a natural isomorphism
        \[ \Hom_{\cat{C}}(A\otimes X,\ Y) \simeq \Hom_{\cat{V}}\bigl(A,\ \Hom_{\cat{C}}(X,Y)\bigr) \]
        for $A\in\cat{V}$ and $X,Y\in\cat{C}$, where $\Hom_{\cat{V}}$ denotes the internal hom of $\cat{V}$.

        \item $\cat{C}$ is \term{cotensored}, or \term{powered}, over $\cat{V}$,
        if there exists a functor
        \[ (-)^{(-)} \: \cat{C} \times \cat{V}\op \to \cat{C}, \]
        with a natural isomorphism
        \[ \Hom_{\cat{C}}(X,\ Y^A) \simeq \Hom_{\cat{V}}\bigl(A,\ \Hom_{\cat{C}}(X,Y)\bigr) \]
        for $A\in\cat{V}$ and $X,Y\in\cat{C}$, where $\Hom_{\cat{V}}$ denotes the internal hom of $\cat{V}$.
    \end{itms}
\end{definition}

\begin{example}
    \ 
    \begin{itms}
        \item As is easily seen, the categories $\cat{Set}$, $\cat{Top}$
        and $\cat{sSet}$ are tensored and cotensored over $\cat{Set}$.

        \item The simplicial categories $\cat{Top}$ and $\cat{sSet}$
        are tensored and cotensored over $\cat{sSet}$.
    \end{itms}
\end{example}

Now we are ready to define a simplicial model category.

\begin{definition}
    Let $\cat{C}$ be a simplicial category.
    The \term{underlying category} $\cat{C}_0$ of $\cat{C}$ is an ordinary category,
    obtained by regarding the $0$-simplices as arrows,
    and discarding all higher simplices.
\end{definition}

\begin{definition}
    A \term{simplicial model category} consists of 
    \begin{itms}
        \item A simplicial category $\cat{C}$.
        \item A model structure on the underlying category $\cat{C}_0$ of $\cat{C}$,
    \end{itms}
    such that
    \begin{itms}
        \item $\cat{C}$ is tensored and cotensored over $\cat{sSet}$.
        \item For every cofibration $i\:A\to B$ in $\cat{sSet}$,
        and any cofibration $j\:X\to Y$ in $\cat{C}_0$,
        the obvious map which we denote by
        \[ i \square j \: A \otimes Y \underset{A\otimes X}{\sqcup} B\otimes X \to B\otimes Y \]
        is a cofibration in $\cat{C}_0$,
        and is trivial whenever $i$ or $j$ is trivial.
    \end{itms}
\end{definition}

The last criterion above may seem somewhat technical,
but it is a very natural requirement if we consider the example 
\[ \begin{aligned}
    \cat{C} &= \cat{V} = \cat{sSet}, \\
    A &= \{0\} \subset \Delta[1], & B &= \Delta[1], \\
    X &= \partial\Delta[2], & Y &= \Delta[2].
\end{aligned} \]
The purpose of this criterion is to ensure that the model structures 
on $\cat{C}$ and on $\cat{V}$ are compatible.

For example, the categories $\cat{Top}$ and $\cat{sSet}$
are simplicial model categories.

An interesting point about simplicial model categories is that 
there are two ways to obtain an $\infty$-category from them.

\begin{itms}
    \item First, a simplicial model category is
    a simplicial category, which gives rise
    to an $\infty$-category.
    \item Second, any category with weak equivalences can be localised 
    to obtain an $\infty$-category.
\end{itms}
Surprisingly, these two $\infty$-categories turn out to be equivalent.
Therefore, in order to study the localised $\infty$-category,
one only needs to look at the simplicial structure of the model category,
which is often quite easy to describe.

\begin{theorem}[Dwyer--Kan]
    Let $\cat{C}$ be a simplicial model category.
    Then $\cat{C}_{\mathrm{cf}}$ is enriched over Kan complexes,
    and there is an equivalence of quasi-categories 
    \[ \cat{C}[\cat{W}^{-1}] \simeq \mathfrak{N}\cat{C}_{\mathrm{cf}}. \]
\end{theorem}

See \cite[Theorem~1.3.4.20]{ha}.

\begin{example}
    As was mentioned in (\ref{rmk-4-locn}),
    this theorem implies that we have equivalences
    \[ \cat S\simeq\cat{sSet}[\cat W^{-1}]\quad\text{and}\quad
    \cat D_\infty(\cat A)\simeq\cat{Ch}_{\cat A}[\cat W^{-1}], \]
    where $\cat{A}$ is an abelian category such that
    $\cat{Ch}_{\cat{A}}$ has the structure of a simplicial model category,
    e.g.\ when $\cat{A}$ is the category of modules over a ring.
    \varqed
\end{example}


